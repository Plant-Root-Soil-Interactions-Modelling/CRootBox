\documentclass[a4paper]{article}

\usepackage[latin1]{inputenc} 
\usepackage{graphicx}
\usepackage{natbib}
\usepackage{amsmath}

\usepackage{listings}
\usepackage{color}

% \usepackage{draftwatermark}
%\usepackage{fullpage}

%\SetWatermarkText{\includegraphics{sw_logo.png}}
\newcommand{\bm}[1]{\mbox{\boldmath{$#1$}}}

\definecolor{codegreen}{rgb}{0,0.6,0}
\definecolor{codegray}{rgb}{0.5,0.5,0.5}
\definecolor{codepurple}{rgb}{0.58,0,0.82}
\definecolor{backcolour}{rgb}{0.95,0.95,0.92}

\lstdefinestyle{mystyle}{
    backgroundcolor=\color{backcolour},   
    commentstyle=\color{codegreen},
    keywordstyle=\color{magenta},
    numberstyle=\tiny\color{codegray},
    stringstyle=\color{codepurple},
    basicstyle=\footnotesize,
    breakatwhitespace=false,         
    breaklines=true,                 
    captionpos=b,                    
    keepspaces=true,                 
    numbers=left,                    
    numbersep=5pt,                  
    showspaces=false,                
    showstringspaces=false,
    showtabs=false,                  
    tabsize=2
}
 
\lstset{style=mystyle}




\begin{document}


\begin{center}
\includegraphics[width=0.2\textwidth]{sw_logo} \\
\vspace{0.5 cm}
\huge{\textbf{CRootBox Tutorial}} \\
\vspace{0.5 cm}
\normalsize
Daniel Leitner \\
www.simwerk.at \\
\end{center}
\vspace{0.5 cm}

\noindent 
The following tutorial offers scripts to outline the usage of the CRootBox Python binding for many different applications. 
For further documentation please refer to the doxygen class documentation of the CRootBox code.

\section*{Basic usage}
 
The first example shows how to use CRootBox in the most simple situation: open a parameter file (L), do the simulation (L), and save the result (L). 

\begin{lstlisting}[language=Python, caption=Example 1a]
import py_rootbox as rb

rootsystem = rb.RootSystem()

# Open plant and root parameter from a file
name = "Anagallis_femina_Leitner_2010" 
rootsystem.openFile(name) 

# Initialize
rootsystem.initialize() 

# Simulate
rootsystem.simulate(30) 

# Export final result (as vtp)
rootsystem.write("results/"+name+".vtp") 
\end{lstlisting}

\noindent 
Lets revise the above code in more detail: 
\begin{itemize}
 \item[L1] Imports Python CRootBox library (py\_rootbox), and names it rb
\end{itemize}






\begin{lstlisting}[language=Python, caption=Example 1b]
import py_rootbox as rb

rootsystem = rb.RootSystem()

# Open plant and root parameter from a file
name = "Anagallis_femina_Leitner_2010" 
rootsystem.openFile(name)

# Create and set geometry

# 1.creates a cylindrical soil core with top radius 5 cm, bot radius 5 cm, height 40cm
soilcore = rb.SDF_PlantContainer(5,5,40,False)

# 2. creates a square 27*27 cm containter with height 1.5 cm
rhizotron = rb.SDF_PlantBox(1.4,27,27);

# 3. creates a square rhizotron r*r, with height h, rotated around the x-axis for angle alpha
r = 20
h = 4
alpha = 45
rhizotron2 = rb.SDF_PlantContainer(r,r,h,True);
posA = rb.Vector3d(0,r,-h/2); #  origin before rotation
A = rb.Matrix3d.rotX(alpha/180.*3.14)
posA = A.times(posA) # origin after rotation
rotatedRhizotron = rb.SDF_RotateTranslate(rhizotron2,alpha,0,posA.times(-1));

# pick 1, 2, or 3
rootsystem.setGeometry(rotatedRhizotron) 

# Initialize
rootsystem.initialize() 

# Simulate
rootsystem.simulate(60)

# Export final result (as vtp)
rootsystem.write("results/"+name+".vtp") 

# Export container geometry as Paraview Python script (run file in Paraview by Tools->Python Shell, Run Script)    
rootsystem.write("results/"+name+".py") 

\end{lstlisting}









 
\section*{More complex geometries}




\section*{Basic analysis}




\section*{Root-soil interaction}





\section*{Extend CRootBox}





\bibliographystyle{apalike}
\bibliography{references}


\end{document}
